\documentclass[12pt]{article}
\usepackage{amsmath}
\usepackage{graphicx}
\usepackage{hyperref}
\usepackage{multirow}
\usepackage{minted}
\usepackage[latin1]{inputenc}

\title{DividendFS}
\author{Superleaf1995}

\begin{document}
\maketitle

\newpage

\section{Preface}
\textbf{DividendFS} (or \textbf{DFS}, as referred in this document for short) is a new filesystem devised from the following ideas:

\begin{itemize}
	\item \textbf{Speed} The filesystem should have a decent speed, both on read, write and modify operations.
	\item \textbf{Ease of implementation} The filesystem should be relatively easy to code for, and be able to have a read-only, no extensions implementation be written in less than 350 bytes.
\end{itemize}

\newpage

\section{Terminology}
\begin{itemize}
	\item Node descriptor: A block used to describe a node.
	\item File fragment: A fragment that in conjunction creates what is formally known as a file.
	\item Journal descriptor: A descriptor that describes a event or action to be done in the filesystem.
	\item File: Refers to the conjuntion of file fragments that conform a file
	\item Children: If used along with node descriptor; Describes the chain of node descriptors which conform a list of children in which the parent is parent of.
	\item Root block: The first block on the root chain, the *NIX equivalent of "/".
\end{itemize}

\newpage

\section{Filesystem}
The pointers in DFS are relative to the start of the header. All numbers and fields are little endian.
Unlike other filesystems, the bootloader does not nescesarily needs to be within the root block (or partition descriptor, in this case), the bootloader can be in it's own 512 byte sector.

\section{Node filesystem}

\subsection{Partition descriptor}

In the first block (boot block or partition start block) of the DFS partition, at the start of the sector a table, described below must be there for making a valid DFS partition:

\begin{minted}{c}
struct dfs_header {
	uint8_t identifier[4];
	uint8_t version;
	uint64_t first_journaling_block;
	uint64_t root_block;
	uint16_t mirror_partition;
	uint16_t bmp_granularity;
	uint64_t size_of_part;
	uint64_t reserved[8];
	uint64_t alloc_bmp[];
};
\end{minted}

\begin{itemize}
	\item \textbf{identifier} Must be '4DFS' (in ASCII) on all implementations.
	\item \textbf{version} Upper nibble denotes major version number, lower nibble denotes lower version number. Current implementation is 03h.
	\item \textbf{first\_journaling\_block} Pointer where the journaling descriptor block resides.
	\item \textbf{root\_block} Pointer where the first root block exists.
	\item \textbf{mirror\_partition} Pointer where the mirror (or backup) DFS header resides at.
	\item \textbf{bmp\_granularity} How many bytes a bit in the bitmap represents.
	\item \textbf{size\_of\_part} Size of partition in 4096 byte units.
	\item \textbf{reserved} Space left for future use. Must always be 0.
	\item \textbf{alloc\_bmp} Array representing the allocation bitmap.
\end{itemize}

\begin{minted}{c}
struct dfs_node_entry {
	uint64_t next_entry;
	uint64_t child_entry;
	uint64_t fragment;
	uint64_t last_fragment;
	uint64_t symlink;
	uint64_t file_size;
	uint64_t reserved[4];
	uint64_t utc_creation_time;
	uint64_t utc_edition_time;
	uint64_t utc_access_time;
	uint16_t nix_perms;
	uint16_t gid;
	uint16_t uid;
	uint64_t reserved[8];
	uint8_t os_id;
	uint64_t os_specific;
	uint32_t name[];
};
\end{minted}

\begin{itemize}
	\item \textbf{child\_entry} Pointer where the first node entry representing the child chain is, if NUL this must be a file/symlink.
	\item \textbf{fragment} Pointer where the first file fragment resides at, if NUL then this must be a directory/symlink.
	\item \textbf{last_fragment} Last file fragment in chain.
	\item \textbf{symlink} Pointer pointing to the symlink target node, NUL meaning this is not a symbolic link.
	\item \textbf{reserved} Reserved fields, set them to 0 for now.
	\item \textbf{utc\_creation\_time} Creation time.
	\item \textbf{utc\_edition\_time} Edition time.
	\item \textbf{utc\_access\_time} Access time.
	\item \textbf{nix_perms} *NIX permissions for group.
	\item \textbf{gid} *NIX id of group.
	\item \textbf{uid} *NIX id of user.
	\item \textbf{reserved} Reserved fields, must be zero.
	\item \textbf{os\_id} OS id who created this node, this allows for OS-specific extensions. An extension can do anything, implementations are not required to know this specific metadata beforehand.
	\item \textbf{os\_specific} OS specific metadata, used per-OS basis.
	\item \textbf{name} A 4NUL terminated string. 4NUL meaning that 4 zeroes represent the end of the string. The string can contain any characters.
\end{itemize}

The following is a table of possible OS id's:
\begin{itemize}
	\item \textbf{0x00} Generic OS, OS Specific is not used.
	\item \textbf{0x01} Means the OS who made this node is a POSIX-compatible OS, the OS Specific denotes the type of node.
	\begin{itemize}
		\item \textbf{0x01} File
		\item \textbf{0x02} Pipe
		\item \textbf{0x03} FIFO pipe
		\item \textbf{0x04} Block device
		\item \textbf{0x05} Symbolic link
		\item \textbf{0x06} Directory symbolic link
		\item \textbf{0x07} Mount point
	\end{itemize}
\end{itemize}

A comforming implementation should not parse a unknown OS ID without proper support. Any OS Specific field should not affect or otherwise break compatibility with other OSes.

If the \textbf{child\_entry} or \textbf{fragment} are not NUL and \textbf{symlink} is not NUL then it's an invalid entry. If all entries are NUL then this entry should be promptly removed.

\begin{minted}{c}
struct dfs_fragment {
	uint64_t size;
	uint64_t checksum;
	uint64_t next;
	uint64_t previous;
	uint8_t data_blob[];
};
\end{minted}

\begin{itemize}
	\item \textbf{size} Size of the fragment entry.
	\item \textbf{checksum} Checksum of file contents (1st byte of data blob XOR'ed by the last byte of the data blob).
	\item \textbf{next} Pointer to next fragment entry.
	\item \textbf{previous} Pointer to previous fragment entry.
	\item \textbf{data\_blob} Raw data blob, no EOF required, take in account size.
\end{itemize}

The pointer to data blob points to the actual data of the file stored in disk. Size tells the size of the data blob. The data blob is always padded to the bitmap granularity. A checksum check is obligatory to keep data in good state.

\newpage

\section{Journaling filesystem}

The Journaling filesystem guarantees log-then-do pipeline when doing file modifications more safely without the worry about lost data or even worse: broken chains. The use of this feature is not required for a compliant implementation.

\begin{minted}{c}
struct dfs_journal_event {
	uint8_t type;
	uint64_t next;
	uint64_t size;
	uint8_t data_blob[];
};
\end{minted}

\begin{itemize}
	\item \textbf{type} Type of journal event.
	\item \textbf{next} Next journal event.
	\item \textbf{size} Size of data blob.
	\item \textbf{data\_blob} Data blob.
\end{itemize}

\newpage

\subsection{Journal event type}

\begin{table}
\centering
\begin{tabular}{ |c|c|c|c| }
\hline
Code & Name & Location & Extra \\
\hline
01h & File modification & Node & Offset \\
02h & Expand Shrink & Node & New size of file \\
03h & Unlink from chain & Node & Previous node \\
04h & Link to chain & Node & Previous node \\
05h & Copy & Node pointer & Parent \\
06h & Change node descriptor & Node & New node \\
07h & Resize & Direction & Offset \\
\hline
\end{tabular}
\caption{Types of journaling events}
\end{table}

\begin{itemize}
	\item \textbf{File modification} Contains data that should replace old data, does not modify size of file, Data should be applied at \textbf{Offset}
	\item \textbf{Expand Shrink} Shrinks or expands a file and makes it's size to be "New size"
	\item \textbf{Unlink from chain} Deletes a node and delinks it from a chain. If \textbf{Previous} node is NUL then the field is ignored
	\item \textbf{Link to chain} Adds a node to a chain. If \textbf{Previous} node is NUL then the field is ignored
	\item \textbf{Copy \& Node pointer} Copies node pointer (and it's children and file fragments) and adds it to the chain of target \textbf{Parent}
	\item \textbf{Change node descriptor \& Node} Replaces the contents of the \textbf{Node} descriptor with a \textbf{New node} descriptor
	\item \textbf{Resize} Forces the driver to make all node descriptors, journal descriptor, file fragments and any other data to be (depending on \textbf{Direction}) above or below \textbf{Offset}, and update the root block accordingly. If direction is 01h then every data must be below, otherwise it must be above
\end{itemize}

\newpage

\section{Sample implementations}
This implementations are given for reference:

\subsection{Reader implementation}
Does not take in account any of the optional extensions of the DFS
\begin{minted}{c}
void * read_file(const char * filename) {
	struct dfs_header * hdr;
	struct dfs_node_entry * node;
	unsigned lba;
	
	hdr = read_disk_at(512); // read sector AFTER bootloader
	
	// read root block
	node = read_disk_at(hdr->root_block);
	
	// while filename not equal we continue recursing
	while(strcmp(node->name,filename)) {
		node = read_disk_at(node->next);
	}
	
	// otherwise we got our node!
	
	// recolect all file fragments
	struct dfs_fragment * frag;
	size_t off;
	void * mem;
	
	mem = malloc(node->size);
	off = 0;
	
	frag = read_disk_at(node->fragment);
	while(1) {
		unsigned char * c, x;
		unsigned checksum;
		
		// place frag data into buffer
		memcpy(mem+off,frag->data_blob,frag->size);
		off += frag->size;
		
		c = (unsigned char *)frag->data_blob;
		x = (unsigned char *)frag->data_blob+frag->size;
		
		// checksum check
		if((*c)^(*x) != frag->checksum)
			kpanic("Checksum check failed!\n");
		
		// go to next fragment
		if(frag->next == NULL)
			break;
		frag = read_disk_at(frag->next);
	}
	
	// return file in memory
	return mem;
}
\end{minted}

\end{document}
